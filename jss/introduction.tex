\newpage
\section{Introduction}

% Defintion of the problem

The term structure of interest rates or the zero-coupon yield curve is the relationship between fixed income investments with only one payment at maturity, and the time to maturity of these cashflows. It is used in various areas of application, e.g. risk management, financial engineering, portfolio management. The zero-coupon yield curve is the basis to value other fixed income instruments. Basically, we can calculate the net present value (NPV) of any cash flow with it. For example, the fair price of a bond is just the sum of its discounted cash flows. By comparing it to the price on the market we can identify mispriced securities.

We give the definitions of a few basic terms used in the fixed income literature. A \emph{discount bond} or \emph{zero-coupon bond} is a fixed income investment with only one payment at maturity. The \emph{spot rate} or \emph{zero-coupon rate} is the interest paid on a discount bond. With continuous compounding the fair price of a discount bond paying \$1 at maturitydate  $m_t$ is given by

\begin{equation}
  \label{eq:pricediscountbond}
  p(m_t)=e^{s(m_t)m_t}
\end{equation}

Therefore, the \emph{zero-coupon yield curve} shows the spot rates for different maturities. The spot rate can also be seen as interest paid for investments that start today ($t=0)$. The \emph{forward rate} $f(t',T)$ is the interest contracted now to be paid for an future investment between time $t'$ and $T$. With continuous compounding we have the following relationship between spot rates and forward rates.

\begin{equation}
  \label{eq:relspotforward}
  e^{s(m_{t'})\cdot m_{t'}}\cdot e^{f(t',T)\cdot (m_T-m_{t'})} = e^{s(m_T)\cdot m_T}
\end{equation}
We can solve (\ref{eq:relspotforward}) for the forward rate.

\begin{equation}
  \label{eq:forwardrate}
  f(t',T) = \frac{s(m_T)\cdot m_T - s(m_{t'})\cdot m_{t'}}{m_T-m_{t'}}
\end{equation}

The \emph{instantaneous forward rate} describes the return for an infinitesmal investment period after the date $t'$.

\begin{equation}
  \label{eq:instforw}
  f(t') = \lim_{T\rightarrow t'}f(t',T)
\end{equation}
\cite{Svensson1994} interprets it as marginal increase in the total return from a marginal increase in the length of the investment period. Therefore, the spot rate can be seen as average of the instantaneous forward rates.

\begin{equation}
  \label{eq:avgforwardrate}
  s(m_t)=\frac{\int_0^{m_t}f(s)ds}{m_t}
\end{equation}

In practice, we can only obtain zero-coupon rates for a limited amount of maturities directly from the market. We have to extract them from \emph{coupon bonds}. The following information is typically observable on the market: the maturity date $m$ , the clean price $p_c$ and the cashflows $c_t$ (coupons and redemption payment) which occur at $t=1,...n$. An investor who wants to buy the bond has to pay the dirty price $p_d$ which consists of the quoted market price (clean price) $p_c$ and the accrued interest $a$.

The term structure of interest rates can usually be described in three ways: the discount function, the
spot rate function and the forward rate function.

The bond pricing equation under continuous compounding is the present value of all cashflows.

\begin{equation}
  \label{bondpriceeq}
  p_c+a = \sum_{t=1}^n \ c_t e^{-s_tm_t}
\end{equation}

The spot rate $s_t$ is the yield-to maturity for a $t$-period zero-coupon bond. $m_t$ is the time to maturity of the $t$-th cash flow. A plot of the yields against times to maturity produces the zero-coupon yield curve.

The yield-to-maturity is the solution for $y$ in the following equation.

\begin{equation}
  \label{yield}
  p_c+a=\sum_{t=1}^n \ c_t e^{-ym_t}
\end{equation}

An equivalent formulation of the bond price equation makes use of the discount factors $d_t=\delta(m_t)=e^{-s_tm_t}$. The continuous discount function $\delta(\cdot)$ is formed by interpolation of the discount factors.


see \cite{Lin2002, Zangari1997}

Prices of discount bonds are not observed, construct such prices by estimating the so-called term structure of interest rates from a set of coupon bonds issued by a government of a corporation.

It is important in various fields of application, e.g. risk management, financial engineering, portfolio management. It can be used to identify mispriced bonds, to price other fixed income securities or interest rate derivatives. Or to calculate the net present value of any type of cash flow. 

Zero-coupon rates can easily obtained from the money market or zero coupon bonds. From coupon bonds it is more complicated. Simplest method is bootstrapping (by whom? \cite{Hagan2006}), only works when all cashflows have the same maturity intervals. Therefore, estimation procedures are needed.

Market segment and bond selection.

Goodness of fit depending on pricing errors import for pricing purposes. Smoothness of the curve.

Fluctuations in the zero-coupon curve are stronger reflected in the forward curve. Smoothness is crucial.

In this section we provide the basic concepts of bond pricing which are required for the estimation of the term structure of interest rates. Section 2 explains the ideas behind the two used models. Details about the notation and the optimisation procedure are presented in section 3. Section 4 includes a complete step-by-step example.


Compounding frequency is the period of time that an investment earns interest. Continous compounding assumes infinitesimally small time amounts.

The zero-coupon rate is the interest paid on a discount bond. A discount bond consist of only one payment at time of maturity.


\begin{equation}
  \label{bondprceq2}
  p_c+a=\sum_{t=1}^n \ c_t \delta(m_t) \end{equation}


The implied $j$-period forward rate is calculated as

\begin{equation}
  \label{forwrate}
  f_{t|j}=\frac{js_j-ts_t}{j-t}=s_j+(s_j-s_t)\frac{t}{j-t}
\end{equation}

The instantaneous forward at the time $t$ equals

\begin{equation}
  \label{instfwdrate}
  f_t=\lim_{\Delta t \rightarrow 0}=\left[s_{t+\Delta t}+(s_{t+\Delta t}-s_t)\frac{t}{\Delta t}\right]=s_t+t\frac{\partial s}{\partial t}
\end{equation}

The spot rate is the integral of the instantaneous forward rate.


Each payment (coupons and principal) has the structure of a discount bond. This makes it possible to relate the coupon bond prices to the spot curve (or zero-coupon yield curve).

\textcolor{red}{\textbf{Literature review}}

\cite{BIS2005, Bolder1999, Bliss1997, Soederlind1997, Ioannides2003, Diebold2006, Nawalkha2005}

\cite{BIS2005} gives an overview of zero-coupon yield curve estimation procedures used by various central banks. The majority uses Nelson-Siegel, Svensson or smoothing splines.

For stylized facts about the yield curve see \cite[p. 7]{Diebold2006}. For different weightings of the Nelson SiegEL see \cite{Soederlind1997, Bliss1997}.

\textcolor{red}{\textbf{Choice of method}}

\cite{Nelson1987, Svensson1994, McCulloch1975}

Price of a discount bond can be modeled as mathematical function, e.g. a cubic spline or an exponential polynomial.

The reason for requiring differentiability of the interpolating function is that then the forward function is continuous \citep[see][p. 97]{Hagan2006}.

The two most popular empirical methodologies for estimating the term structure of interest rates are.

Two categories of methodologies for the term structure of interest rates.

Goal is to fit the data and obtain a smooth curve.

\textcolor{red}{\textbf{Principal results}}

The \proglang{R} package \pkg{termstrc} consists of methods for estimating the term structure of interest rates from market data.

\textcolor{red}{\textbf{Conclusions}}

\cite{Filipovic1999, Bjoerk1999} show that there does not exist any nontrivial interest rate model that is consistent with the Nelson-Siegel family. \cite{Filipovic2000} says that the Nelson-Siegel and Svensson families may be well suited for daily estimations of the forward curve, but they are best not used for intertemporal interest rate modeling by diffusion processes. \cite{Krippner2006} derives and intertemporally consistent and arbitrage-free version of the Nelson-Siegel model. It can be used for forecasting the yieldcurve.

For a practical implementation see \cite{Bolder2006}. 

%%% Local Variables: 
%%% mode: latex
%%% TeX-master: "jss-termstrc"
%%% End: 
