\newpage
\section{Introduction}

\begin{itemize}
\item importance of term structure estimation, \cite{BIS2005, Bolder1999, Bliss1997, Soenderlind1997 Ioannides2003, Diebold2006, Nawalkha2005}
\item no-arbitrage and equilibrium models 
\item bootstrapping
\end{itemize}

\cite{BIS2005}

For stylized facts about the yield curve see \cite[p. 7]{Diebold2006}. For different weightings of the Nelson Siegle see \cite{Soederlind1997, Bliss1997}.

The \proglang{R} package \pkg{termstrc} consists of methods for estimating the term structure of interest rates from market data.

In this section we provide the basic concepts of bond pricing which are required for the estimation of the term structure of interest rates. Section 2 explains the ideas behind the two used models. Details about the notation and the optimisation procedure are presented in section 3. Section 4 includes a complete step-by-step example.


Since zero-coupon rates are rarely directly observable, they have to be estimated from market data for existing coupon bonds. For a coupon bond the following information is typically observable on the market: the maturity date $m$ , the clean price $p_c$ and the cashflows $c_t$ (coupons and redemption payment) which occur at $t=1,...n$. An investor who wants to buy the bond has to pay the dirty price $p_d$ which consists of the quoted market price (clean price) $p_c$ and the accrued interest $a$.

The term structure of interest rates can usually be described in three ways: the discount function, the
spot rate function and the forward rate function.

The bond pricing equation under continuous compounding is the present value of all cashflows.

\begin{equation}
  \label{bondpriceeq}
  p_c+a = \sum_{t=1}^n \ c_t e^{-s_tm_t}
\end{equation}

The spot rate $s_t$ is the yield-to maturity for a $t$-period zero-coupon bond. $m_t$ is the time to maturity of the $t$-th cash flow. A plot of the yields against times to maturity produces the zero-coupon yield curve.

The yield-to-maturity is the solution for $y$ in the following equation.

\begin{equation}
  \label{yield}
  p_c+a=\sum_{t=1}^n \ c_t e^{-ym_t}
\end{equation}

An equivalent formulation of the bond price equation makes use of the discount factors $d_t=\delta(m_t)=e^{-s_tm_t}$. The continuous discount function $\delta(\cdot)$ is formed by interpolation of the discount factors.

\begin{equation}
  \label{bondprceq2}
  p_c+a=\sum_{t=1}^n \ c_t \delta(m_t) \end{equation}


The implied $j$-period forward rate is calculated as

\begin{equation}
  \label{forwrate}
  f_{t|j}=\frac{js_j-ts_t}{j-t}=s_j+(s_j-s_t)\frac{t}{j-t}
\end{equation}

The instantaneous forward at the time $t$ equals

\begin{equation}
  \label{instfwdrate}
  f_t=\lim_{\Delta t \rightarrow 0}=\left[s_{t+\Delta t}+(s_{t+\Delta t}-s_t)\frac{t}{\Delta t}\right]=s_t+t\frac{\partial s}{\partial t}
\end{equation}

The spot rate is the integral of the instantaneous forward rate.




%%% Local Variables: 
%%% mode: latex
%%% TeX-master: "jss-termstrc"
%%% End: 
