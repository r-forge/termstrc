\section{Zero-coupon yield curve estimation}

\subsection{Notation}
\label{sec:notation}

Let us establish the necessary notation for a market data set of coupon bonds.

\subsubsection*{Maturity matrix}

\begin{equation}\label{maturitym}
\bm{M}_{\left[n\times m\right]}= \{m_{ij}\}
\end{equation}

The number of rows $n$ is determined through the number of cashflows of the bond $j$ with the longest maturity. For each bond $j$ exists a column with the corresponding cashflow dates. Dates after the maturity of the bond $j$ are filled up with zeros till the maturity date of the bond with the longest maturity. One element $m_{ij}$ of the matrix  refers, therefore, to the maturity date of  the $i$-th cashflow of the $j$-th bond. 

\subsubsection*{Maturity vector}

We denote with $m_j$ the maturity of the last cashflow, i.e. the maturity of the $j$-th bond.

\begin{equation}\label{weights}
    \bm{m}_{\left[1\times m\right]}= \{m_j\}
\end{equation}

\subsubsection*{Cashflow matrix}

 \begin{equation}\label{cashflowm}
\bm{C}_{\left[n\times m\right]}= \{c_{ij}\}
\end{equation}

 The cashflow matrix is defined analogously to the maturity matrix.  One element $c_{ij}$  of the matrix refers to the $i$-th cashflow of the $j$-th bond. Note, that the last cashflow of a each bond includes the redemption payment.

\subsubsection*{Discount factor matrix}

 \begin{equation}\label{discountm}
\bm{D}_{\left[n\times m\right]}= \{d_{ij}\}
\end{equation}

The discount factor matrix is also defined analogously to the maturity matrix. One element $d_{ij}$ of the matrix refers to the discount factor associated with  the $i$-th cashflow of the $j$-th bond. The discount function $d(m_{i,j})$ returns the discount factor for a given maturity. We will see in the following sections several methods how to estimate it. From an economic point of view only positve interest rates make sense. This implies that the discount factors are nonnegative where the entries in the maturity matrix are greater zero. Remember, zero entries in the maturity matrix mean that for these points in time now cash flows are associated.

\subsubsection*{Clean price vector}

 \begin{equation}\label{pc}
\bm{p}^c_{\left[1\times m\right]}= \{p^c_j\}
\end{equation}

$p_{j}^c$ is the quoted price of the $j$-th bond.

\textcolor{red}{Some blabla about price quotation on the market.}

\subsubsection*{Accrued interest vector}

  \begin{equation}\label{a}
\bm{a}_{\left[1\times m\right]}= \{a_j\}
\end{equation}


\textcolor{red}{Motivate need to calculate accrued interest. Buyer gets coupon payment but has not held bond for the whole coupon period $\dots$}

Different conventions for the calculation of accrued interest are used in the market. A basic form for the $j$-th bond is as follows.

\begin{equation}
    a_j= \frac{\mbox{number of days since last coupon payment}}{\mbox{number of days in current coupon period}}\cdot \mbox{coupon}_j
\end{equation}
 	

\subsubsection*{Dirty price vector}

The dirty price vector is the sum of the clean price vector and the accrued interest vector and consists of the dirty prices of all bonds $j$.

\begin{displaymath}
\bm{p}=\bm{p}^c+\bm{a}
\end{displaymath}

The elements are denoted by 
\begin{equation}\label{pd}
    \bm{p}_{\left[1\times m\right]}= \{p_j\}\,.
\end{equation}


\subsubsection*{Weights matrix}

In section \ref{sec:nels-svenss-meth} we will use a weighting for the estimation errors. It is constructed  as follows.

\begin{equation}\label{weights}
    \bm{\Omega}_{\left[m\times m\right]}= \begin{pmatrix}
 \omega_1 & 0 &\cdots  &0  \\
 0 & \omega_2 &  & \vdots \\
 \vdots &  & \ddots & 0 \\
 0 &\cdots  &0  & \omega_m
\end{pmatrix}
\end{equation}

Whereas $\omega_j$ is the weight for bond $j$ with duration $d_j$:

\begin{displaymath}
    \omega_j=\frac{\frac{1}{d_j}}{\sum_{i=1}^m\frac{1}{d_i}}
\end{displaymath}


The duration for a bond $j$ is a weighted average of the time to cashflows.

\textcolor{red}{Notation unclear, we would need the $j$-th column from $\bm{C,D,M}$.}

\begin{equation}\label{duration}
d_j= \frac{\bm{C}_{\left[n \times j\right]} \left(\bm{D}_{\left[n\times j\right]} \cdot \bm{M}_{\left[n\times j\right]}\right)^{\top}} {\bm{C}_{\left[n \times j\right]}\left(\bm{D}_{\left[n\times j\right]}\right)^{\top}}
\end{equation}



For the rest of the paper $(\cdot)$ denotes a element by element multiplication and $( )'$ the transposed matrix (vector). $\bm{\iota}$ denotes a column vector filled with ones.

\subsection{Indirect estimation procedure}
\label{sec:estimation}

The theoretical prices are defined as sums of the discounted cash flows of each bond.

\begin{equation}
  \label{eq:theorprices}
  \bm{\hat{p}} = \bm{\iota}'(\bm{C}\cdot\bm{D})
\end{equation}



The pricing errors
\begin{equation}
  \label{eq:pricingerrors}
  \bm{\epsilon} = \bm{p-\hat{p}}
\end{equation}

are the deviation of the theoretical prices from the dirty prices observed on the market. They satisfy

\begin{eqnarray}
  \label{eq:residuals}
  \E(\bm{\epsilon}) &=& \bm{0}\\
  \VAR(\bm{\epsilon}) &=&\sigma^2\bm{\Omega}^2 \\
  \COV(\epsilon_i, \epsilon_j) &=& 0 \quad \mbox{for}\,i\neq j\quad .
\end{eqnarray}

The goal is to minimize the weighted squared pricing errors.


\begin{equation}
  \label{eq:optimalparam}
  \hat{\bm{b}}= \argmin_{\bm{b}}\bm{\iota}'(\bm{\Omega}\bm{\epsilon}^2)
\end{equation}

Goodness of fit can be measured with the root mean squared error

\begin{equation}
  \label{eq:rmse}
  \mbox{RMSE}=\sqrt{\frac{1}{m}\bm{\iota}'(\bm{\Omega\epsilon}^2)}
\end{equation}
and the mean absolute error

\begin{equation}
  \label{eq:mae}
  \mbox{MAE}=\frac{1}{m}\bm{\iota}'|\bm{\epsilon}|\quad.
\end{equation}


%%% Local Variables: 
%%% mode: latex
%%% TeX-master: "jss-termstrc"
%%% End: 
