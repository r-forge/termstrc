
\section{Notation}
\label{sec:notation}

\subsubsection*{Maturity matrix $\bm{M}$}

\begin{equation}\label{maturitym}
\bm{M}_{\left[n\times m\right]}:= \{m_{ij}\}
\end{equation}

The number of rows $n$ is determined through the number of cashflows of the bond $j$ with the longest maturity. For each bond $j$ exists a column with the corresponding cashflow dates. Dates after the maturity of the bond $j$ are filled up with zeros till the maturity date of the bond with the longest maturity. One element $m_{ij}$ of the matrix  refers, therefore, to the maturity date of  the $i$-th cashflow of the $j$-th bond. We denote with $m_j$ the maturity of the last cashflow, i.e. the maturity of the $j$-th bond.

\begin{equation}\label{weights}
    \bm{m}_{\left[1\times m\right]}:= \{m_j\}
\end{equation}

\subsubsection*{Cashflow matrix $\bm{C}$}

 \begin{equation}\label{cashflowm}
\bm{C}_{\left[n\times m\right]}:= \{c_{ij}\}
\end{equation}

 The cashflow matrix is defined analogously to the maturity matrix.  One element $c_{ij}$  of the matrix refers to the $i$-th cashflow of the $j$-th bond. Note, that the last cashflow of a each bond includes the redemption payment.

\subsubsection*{Discount factor matrix $\bm{D}$}

 \begin{equation}\label{discountm}
\bm{D}_{\left[n\times m\right]}:= \{d_{ij}\}
\end{equation}

 The discount factor matrix is also defined analogously to the maturity matrix. One element $d_{ij}$ of the matrix refers to the discount factor associated with  the $i$-th cashflow of the $j$-th bond.

\subsubsection*{Clean price vector $\bm{p}^c$}

  \begin{equation}\label{pc}
\bm{p}^c_{\left[1\times m\right]}:= \{p^c_j\}
\end{equation}

$p_{c_j}$ is the quoted price of the $j$-th bond ($j=1...m$).

\subsubsection*{Accrued interest vector $\bm{a}$}

  \begin{equation}\label{a}
\bm{a}_{\left[1\times m\right]}:= \{a_j\}
\end{equation}

Different conventions for the calculation of accrued interest are used in the market. A basic form for the $j$-th bond is as follows.

\begin{equation}
    a_j= \frac{\mbox{number of days since last coupon payment}}{\mbox{number of days in current coupon period}}\cdot \mbox{coupon}_j
\end{equation}
 	

\subsubsection*{Dirty price vector $\bm{p}^d$}

\begin{equation}\label{pd}
    \bm{p}^d_{\left[1\times m\right]}:= \{p^d_j\}
\end{equation}

The dirty price vector is the sum of the clean price vector and the accrued interest vector and consists of the dirty prices of all bonds $j$.

\begin{displaymath}
\bm{p}^d=\bm{p}^c+\bm{a}
\end{displaymath}

\subsubsection*{Weights vector $\bm{w}$}

\begin{equation}\label{weights}
    \bm{w}_{\left[1\times m\right]}:= \{w_j\}
\end{equation}

Whereas $\omega_j$ is the weight for bond $j$ with Duration $d_j$:

\begin{displaymath}
    w_j=\frac{\frac{1}{d_j}}{\sum_{i=1}^m\frac{1}{d_i}}
\end{displaymath}


The duration for a bond $j$ is a weighted average of the time to cashflows.
%old definition
%\begin{equation}
  %\label{duration}
 % D=\frac{C\sum_{i=1}^n\delta(m_i)m_i+\delta(m_n)Rm_n}{C\sum_{i=1}^n\delta(m_i)+\delta(m_n)R}=\frac{1}{p_c+a}\left[C\sum_{i=1}^n\delta(m_i)m_i+\delta(m_n)Rm_n\right]
%\end{equation}

\begin{equation}\label{duration}
d_j= \frac{\bm{C}_{\left[n \times j\right]} \left(\bm{D}_{\left[n\times j\right]} \cdot \bm{M}_{\left[n\times j\right]}\right)^{\top}} {\bm{C}_{\left[n \times j\right]}\left(\bm{D}_{\left[n\times j\right]}\right)^{\top}}
\end{equation}

Whereas $(\cdot)$ denotes a element by element multiplication and $( )^{\top}$ the transposed matrix (vector).



\begin{equation}\label{eq:weights}
  \bm{w}_{\left[1\times m\right]}= \{w_j\}; \qquad   w_j=\frac{\frac{1}{D_j}}{\sum_{i=1}^m\frac{1}{D_i}}
\end{equation}


%%% Local Variables: 
%%% mode: latex
%%% TeX-master: "jss-termstrc"
%%% End: 
