
\section{Nelson/Siegel and Svensson method}
\label{sec:nels-svenss-meth}


\begin{itemize}
\item \cite{Geyer1999}
\item duration based weights \cite{Bliss1997}
\item derivation via Laguerre function
\end{itemize}

\subsubsection*{Maturity matrix $\bm{M}$}

\begin{equation}\label{maturitym}
\bm{M}_{\left[n\times m\right]}:= \{m_{ij}\}
\end{equation}

The number of rows $n$ is determined through the number of cashflows of the bond $j$ with the longest maturity. For each bond $j$ exists a column with the corresponding cashflow dates. Dates after the maturity of the bond $j$ are filled up with zeros till the maturity date of the bond with the longest maturity. One element $m_{ij}$ of the matrix  refers, therefore, to the maturity date of  the $i$-th cashflow of the $j$-th bond.

\subsubsection*{Cashflow matrix $\bm{C}$}

 \begin{equation}\label{cashflowm}
\bm{C}_{\left[n\times m\right]}:= \{c_{ij}\}
\end{equation}

 The cashflow matrix is defined analogously to the maturity matrix.  One element $c_{ij}$  of the matrix refers to the $i$-th cashflow of the $j$-th bond. Note, that the last cashflow of a each bond includes the redemption payment.

\subsubsection*{Discount factor matrix $\bm{D}$}

 \begin{equation}\label{discountm}
\bm{D}_{\left[n\times m\right]}:= \{d_{ij}\}
\end{equation}

 The discount factor matrix is also defined analogously to the maturity matrix. One element $d_{ij}$ of the matrix refers to the discount factor associated with  the $i$-th cashflow of the $j$-th bond. An individual discount factor $d_{ij}$ is constructed as

\begin{displaymath}
d_{ij}=e^{-m_{ij}s(m_{ij},b)},
\end{displaymath}

where $s(m_{ij},b)$ is the Nelson/Siegel or Svensson spot rate function defined in equation (\ref{nsspot}) and (\ref{svspot}).

\subsubsection*{Clean price vector $\bm{p}^c$}

  \begin{equation}\label{pc}
\bm{p}^c_{\left[1\times m\right]}:= \{p^c_j\}
\end{equation}

$p_{c_j}$ is the quoted price of the $j$-th bond ($j=1...m$).

\subsubsection*{Accrued interest vector $\bm{a}$}

  \begin{equation}\label{a}
\bm{a}_{\left[1\times m\right]}:= \{a_j\}
\end{equation}

Different conventions for the calculation of accrued interest are used in the market. A basic form for the $j$-th bond is as follows.

\begin{equation}
    a_j= \frac{\mbox{number of days since last coupon payment}}{\mbox{number of days in current coupon period}}\cdot \mbox{coupon}_j
\end{equation}
 	

\subsubsection*{Dirty price vector $\bm{p}^d$}

\begin{equation}\label{pd}
    \bm{p}^d_{\left[1\times m\right]}:= \{p^d_j\}
\end{equation}

The dirty price vector is the sum of the clean price vector and the accrued interest vector and consists of the dirty prices of all bonds $j$.

\begin{displaymath}
\bm{p}^d=\bm{p}^c+\bm{a}
\end{displaymath}

\subsubsection*{Weights vector $\bm{w}$}

\begin{equation}\label{weights}
    \bm{w}_{\left[1\times m\right]}:= \{w_j\}
\end{equation}

Whereas $\omega_j$ is the weight for bond $j$ with Duration $d_j$:

\begin{displaymath}
    w_j=\frac{\frac{1}{d_j}}{\sum_{i=1}^m\frac{1}{d_i}}
\end{displaymath}


The duration for a bond $j$ is a weighted average of the time to cashflows.
%old definition
%\begin{equation}
  %\label{duration}
 % D=\frac{C\sum_{i=1}^n\delta(m_i)m_i+\delta(m_n)Rm_n}{C\sum_{i=1}^n\delta(m_i)+\delta(m_n)R}=\frac{1}{p_c+a}\left[C\sum_{i=1}^n\delta(m_i)m_i+\delta(m_n)Rm_n\right]
%\end{equation}

\begin{equation}\label{duration}
d_j= \frac{\bm{C}_{\left[n \times j\right]} \left(\bm{D}_{\left[n\times j\right]} \cdot \bm{M}_{\left[n\times j\right]}\right)^t} {\bm{C}_{\left[n \times j\right]}\left(\bm{D}_{\left[n\times j\right]}\right)^t}
\end{equation}

Whereas $(\cdot)$ denotes a element by element multiplication and $( )^t$ the transposed matrix (vector).

\cite{Nelson1987} propose a parsimonious  model of  the instantaneous forward rate as a solution to a second-order differential equation for the case of equal roots.

\begin{equation}
  \label{eq:laguerre}
  f(m,\bm{b}) = \beta_0+\beta_1\exp\left(-\frac{m}{\tau_1}\right)+\beta_2\left[\left(\frac{m}{\tau_1}\right)\exp\left(-\frac{m}{\tau_1}\right)\right]
\end{equation}


The spot rate is the average of the instantaneous forward rates. 

\begin{equation}
  \label{eq:intspotrate}
  s(m,\bm{b})=\frac{1}{m}\int_0^mf(m,\bm{b})\,dm
\end{equation}


\begin{equation}
  \label{eq:nelson-spot}
   s(m,\bm{b}) = \beta_0 + \beta_1\frac{1-\exp(-\frac{m}{\tau_1})}{\frac{m}{\tau_1}} + \beta_2\left(\frac{1-\exp(-\frac{m}{\tau_1})}{\frac{m}{\tau_1}} - \exp(-\frac{m}{\tau_1})\right)
\end{equation}

\input{curveshape.tex}

%\input{curveflex.tex}
 
\cite{Svensson1994} extension


\begin{multline}\label{eq:svensson-spot}
    s(m,\bm{b}) = \beta_0 + \beta_1\frac{1-\exp(-\frac{m}{\tau_1})}{\frac{m}{\tau_1}} + \beta_2\left(\frac{1-\exp(-\frac{m}{\tau_1})}{\frac{m}{\tau_1}} - \exp(-\frac{m}{\tau_1})\right) \\+ \beta_3\left(\frac{1-\exp(-\frac{m}{\tau_2})}{\frac{m}{\tau_2}} - \exp(-\frac{m}{\tau_2})\right)
\end{multline}



with a parameter vector ${\bm{b}} = \left(\beta_0,\beta_1,\beta_2,\tau_1,\beta_3,\tau_2\right)$.

The impact of the parameters can be described as follows \citep[see][p.7]{Bolder1999}:

\begin{itemize}
\item $\beta_0$ is the asymptotic value of the forward rate (spot rate) function.  The curve will tend towards the asymptote as the time to maturity approaches  $\infty$ ($\beta_0 >0$).
\item $\beta_1$ determines the starting (short-term) value of the curve in terms of deviation from the asymptote (the sum $\beta_0$ and $\beta_1$) is the vertical intercept. Moreover, it defines the basic speed with which the curve tends toward its long-term trend.
\item $\tau_1$ specifies the position of the first hump or the U-shape on the curve ($\tau_1>0$).
\item $\beta_2$ determines the magnitude and direction of the hump. If $\beta_2 >0$  a hump will occur at  $\tau_1$, whereas $\beta_2<0$, a U-shaped value will occur at $\tau_1$.
\item $\tau_2$ specifies the position of the second hump or the U-shape on the curve ($\tau_2>0$).
\item $\beta_3$ analogously  to  $\beta_2$ determines the magnitude and direction of the second hump.
\end{itemize}



\begin{equation}\label{eq:weights}
  \bm{w}_{\left[1\times m\right]}= \{w_j\}; \qquad   w_j=\frac{\frac{1}{D_j}}{\sum_{i=1}^m\frac{1}{D_i}}
\end{equation}


\begin{equation}\label{eq:objfct-nelson}
	F=\left( \bm{\iota}_{\left[1 \times n\right]}\left[\bm{C}\cdot\bm{D}\right] - 		\bm{p}^d\right)^2 \bm{w}\bm{\iota}_{\left[m \times 1\right]}
\end{equation}





%%% Local Variables: 
%%% mode: latex
%%% TeX-master: "jss-termstrc"
%%% End: 
