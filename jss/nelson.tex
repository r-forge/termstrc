\newpage
\section{Nelson/Siegel and Svensson method}
\label{sec:nels-svenss-meth}


\begin{itemize}
\item \cite{Geyer1999}
\item duration based weights \cite{Bliss1997}
\item derivation via Laguerre function
\item extensions \cite{Bjoerk 1999, Filipovic1999, Filipovic2000, Bjoerk2001, Bjoerk2002, Soederlind1997, Bliss1997}
\end{itemize}


\cite{Nelson1987} propose a parsimonious  model of  the instantaneous forward rate as a solution to a second-order differential equation for the case of equal roots.

\begin{equation}
  \label{eq:laguerre}
  f(m,\bm{b}) = \beta_0+\beta_1\exp\left(-\frac{m}{\tau_1}\right)+\beta_2\left[\left(\frac{m}{\tau_1}\right)\exp\left(-\frac{m}{\tau_1}\right)\right]
\end{equation}


The spot rate is the average of the instantaneous forward rates. 

\begin{equation}
  \label{eq:intspotrate}
  s(m,\bm{b})=\frac{1}{m}\int_0^mf(m,\bm{b})\,dm
\end{equation}


\begin{equation}
  \label{eq:nelson-spot}
   s(m,\bm{b}) = \beta_0 + \beta_1\frac{1-\exp(-\frac{m}{\tau_1})}{\frac{m}{\tau_1}} + \beta_2\left(\frac{1-\exp(-\frac{m}{\tau_1})}{\frac{m}{\tau_1}} - \exp(-\frac{m}{\tau_1})\right)
\end{equation}

\input{curveshape.tex}

%\input{curveflex.tex}
 
\cite{Svensson1994} extension


\begin{multline}\label{eq:svensson-spot}
    s(m,\bm{b}) = \beta_0 + \beta_1\frac{1-\exp(-\frac{m}{\tau_1})}{\frac{m}{\tau_1}} + \beta_2\left(\frac{1-\exp(-\frac{m}{\tau_1})}{\frac{m}{\tau_1}} - \exp(-\frac{m}{\tau_1})\right) \\+ \beta_3\left(\frac{1-\exp(-\frac{m}{\tau_2})}{\frac{m}{\tau_2}} - \exp(-\frac{m}{\tau_2})\right)
\end{multline}



with a parameter vector ${\bm{b}} = \left(\beta_0,\beta_1,\beta_2,\tau_1,\beta_3,\tau_2\right)$.

The impact of the parameters can be described as follows \citep[see][p.7]{Bolder1999}:

\begin{itemize}
\item $\beta_0$ is the asymptotic value of the forward rate (spot rate) function.  The curve will tend towards the asymptote as the time to maturity approaches  infinity ($\beta_0 >0$).
\item $\beta_1$ determines the starting (short-term) value of the curve in terms of deviation from the asymptote (the sum $\beta_0$ and $\beta_1$) is the vertical intercept. Moreover, it defines the basic speed with which the curve tends toward its long-term trend.
\item $\tau_1$ specifies the position of the first hump or the U-shape on the curve ($\tau_1>0$).
\item $\beta_2$ determines the magnitude and direction of the hump. If $\beta_2 >0$  a hump will occur at  $\tau_1$, whereas $\beta_2<0$, a U-shaped value will occur at $\tau_1$.
\item $\tau_2$ specifies the position of the second hump or the U-shape on the curve ($\tau_2>0$).
\item $\beta_3$ analogously  to  $\beta_2$ determines the magnitude and direction of the second hump.
\end{itemize}


The discount factor for any maturity can be calculated as follows. 

\begin{displaymath}
d(m_{ij})=e^{-m_{ij}s(m_{ij},b)},
\end{displaymath}

where $s(m_{ij},b)$ is the Nelson/Siegel or Svensson spot rate function defined in \eqref{eq:intspotrate} and \eqref{eq:nelson-spot}.

Theoretical prices are defined as

\begin{equation}
  \label{eq:theorprices}
  \bm{p}_{[1\times m]} = \bm{\iota}_{[1\times n]}\bm{C}_{[n\times m]}\cdot\bm{D}_{[n\times m]}+\bm{\epsilon}_{[1\times m]}
\end{equation}

where $\bm{\iota}$ is a vetor of ones and  $\bm{\epsilon}$ is a vector of idiosyncratic errors.

The goal is to minimize the squared pricing errors, which can also be weighted.

\begin{equation}\label{eq:objfct-nelson}
	F=\left(\bm{p} - \bm{\hat{p}}\right)^2 \bm{w}\bm{\iota}_{\left[m \times 1\right]}
\end{equation}

\begin{equation}
  \label{eq:objfct-nelsonmin}
  \argmin_{\bm{b} \in G} F(\bm{b})
\end{equation}

box-constraints

	
	 $$ \bm{b} \in G =   \{ \bm{b} \in \mathbb{R}^4 ; \mathbb{R}^6 : \mathbf{lb} \leq \bm{b} \leq \mathbf{ub}  \}$$



%%% Local Variables: 
%%% mode: latex
%%% TeX-master: "jss-termstrc"
%%% End: 
