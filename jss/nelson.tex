\section{Nelson/Siegel and Svensson method}
\label{sec:nels-svenss-meth}

\cite{Nelson1987} propose a parsimonious  model of  the instantaneous forward rate as a solution to a second-order differential equation for the case of equal roots

\begin{equation}
  \label{eq:laguerre}
  f(m_{ij},\bm{\beta}) = \beta_0+\beta_1\exp\left(-\frac{m_{ij}}{\tau_1}\right)+\beta_2\left[\left(\frac{m_{ij}}{\tau_1}\right)\exp\left(-\frac{m_{ij}}{\tau_1}\right)\right]\,,
\end{equation}

with a parameter vector ${\bm{\beta}} = \left(\beta_0,\beta_1,\beta_2,\tau_1\right)$. As in \eqref{eq:avgforwardrate}, the spot rate is the average of the instantaneous forward rates

\begin{equation*}
  \label{eq:intspotrate}
  s(m_{ij},\bm{\beta})=\frac{1}{m_{ij}}\int_0^{m_{ij}}f(m_{ij},\bm{\beta})\,dm_{ij}\,,
\end{equation*}

resulting in

\begin{equation}
  \label{eq:nelson-spot}
   s(m_{ij},\bm{\beta}) = \beta_0 + \beta_1\frac{1-\exp(-\frac{m_{ij}}{\tau_1})}{\frac{m_{ij}}{\tau_1}} + \beta_2\left(\frac{1-\exp(-\frac{m_{ij}}{\tau_1})}{\frac{m_{ij}}{\tau_1}} - \exp\left(-\frac{m_{ij}}{\tau_1}\right)\right).
\end{equation}

 
This specifiation can produce a wide range of possible curve shapes, including monotonic, humped, $U$-shapes or $S$-shapes. \cite{Svensson1994} adds another term with two new parameters to increase the flexibility. It allows for a second hump in the curve. The spot rate function is then defined as


\begin{multline}\label{eq:svensson-spot}
    s(m_{ij},\bm{\beta}) = \beta_0 + \beta_1\frac{1-\exp(-\frac{m_{ij}}{\tau_1})}{\frac{m_{ij}}{\tau_1}} + \beta_2\left(\frac{1-\exp(-\frac{m_{ij}}{\tau_1})}{\frac{m_{ij}}{\tau_1}} - \exp\left(-\frac{m_{ij}}{\tau_1}\right)\right) \\+ \beta_3\left(\frac{1-\exp(-\frac{m_{ij}}{\tau_2})}{\frac{m_{ij}}{\tau_2}} - \exp\left(-\frac{m_{ij}}{\tau_2}\right)\right)\,,
\end{multline}



with a parameter vector ${\bm{\beta}} = \left(\beta_0,\beta_1,\beta_2,\tau_1,\beta_3,\tau_2\right)$. Figure \ref{fig:modelcurves} shows the \cite{Svensson1994} spot rate function and the impact of the different components using $\bm{\beta} = (1, 1, 4, 5, 1, 10)$. The parameters have the following interpretations:

%\input{curveshape.tex}

\begin{figure}[htb]
\centering
  \caption{Model curves}
 \label{fig:modelcurves}
\includegraphics[width=0.7\textwidth]{curveshape}
\end{figure}

\begin{itemize}
\item $\beta_0>0$ is the asymptotic value of the spot rate function $\lim_{m_{ij}\to\infty}s(m_{ij},\bm{\beta})$, which can be seen as the long-term interest rate.
\item $\beta_1$ determines the rate of convergence with which the spot rate function approaches its long-term trend, and $\beta_0+\beta_1$ is the starting value of the curve at the short end. The slope will be negative if $\beta_1>0$ and vice versa.
\item $\beta_2$ determines the size and the form of the hump. $\beta_2 >0$  results in a hump at  $\tau_1$, whereas $\beta_2<0$ produces a $U$-shape.
\item $\tau_1>0$ specifies the location of the first hump or the $U$-shape on the curve.
\item $\beta_3$, analogously to $\beta_2$, determines the size and form of the second hump.
\item $\tau_2>0$ specifies the position of the second hump.
\end{itemize}

The discount factor for any maturity $m_{ij}$ can be calculated as follows:

\begin{displaymath}
\delta(m_{ij}, \bm{\beta})=e^{-m_{ij}s(m_{ij},\bm{\beta})}\,,
\end{displaymath}

where $s(m_{ij},\bm{\beta})$ is the Nelson/Siegel or Svensson spot rate function defined in \eqref{eq:nelson-spot} and \eqref{eq:svensson-spot}. We optimize the objective function in \eqref{eq:optimalparam}. The above specification of the discount function leads to a nonlinear optimization problem. Good starting values for the parameter vector are important to find a global minimum. Instead of minimizing pricing errors, it is common to minimize the yield errors. The yield-to-maturities of the theoretical bond prices can easily be calculated numerically, because \eqref{eq:yield} has only one real root. 

When minimizing the unweighted price errors, bonds with a longer maturity get a higher weighting, because they are more sensitive to changes in prices, which leads to a less accurate fit at the short end. Solutions to this heteroskedasticity problem are to use weights for the pricing errors, or to minimize the yield errors. A possible specification for the weights is based on the inverse of the duration as in \eqref{eq:durationweight} \citep[see][]{Bliss1997}.


%%% Local Variables: 
%%% mode: latex
%%% TeX-master: "jss-termstrc"
%%% End: 
