\documentclass[article]{jss}
\usepackage{amsmath,bm}
%\usepackage{pdfdraftcopy}


%%%%%%%%%%%%%%%%%%%%%%%%%%%%%%
%% declarations for jss.cls %%%%%%%%%%%%%%%%%%%%%%%%%%%%%%%%%%%%%%%%%%
%%%%%%%%%%%%%%%%%%%%%%%%%%%%%%

%% almost as usual
\author{Robert Ferstl\\Wirtschaftsuniversit\"at Wien \And 
        Josef Hayden\\ Wirtschaftsuniversit\"at Wien}
\title{Zero-Coupon Yield Curve Estimation\\ with the Package \pkg{termstrc}}

%% for pretty printing and a nice hypersummary also set:
\Plainauthor{Robert Ferstl, Josef Hayden} %% comma-separated
\Plaintitle{Zero-Coupon Yield Curve Estimation with the Package termstrc} %% without formatting
\Shorttitle{Zero-Coupon Yield Curve Estimation with the Package termstrc} %% a short title (if necessary)

%% an abstract and keywords
\Abstract{
Zero-coupon yield curves and credit spread curves are important inputs for various financial models, e.g. pricing of securities, risk management, monetary policy issues. Since zero-coupon rates are rarely directly observable, they have to be estimated from market data, e.g. of existing coupon bonds. The literature broadly distinguishes between parametric and spline-based estimation methods for the zero-coupon yield curve. Our package consists of several widely-used approaches, i.e. the parametric \cite{Nelson1987} method with the \cite{Svensson1994} extension, and the \cite{McCulloch1971} cubic splines approach. Moreover, we implement the traditional way of credit spread calculation, where individually estimated zero-coupon yield curves are substracted from a risk-free reference curve. Goodness-of-fit tests are provided to compare the results of the different estimation methods. We illustrate the usage of our functions by practical examples with data from European and CEE government bonds, and European corporate bonds.
}
\Keywords{fixed income, term structure estimation, credit spreads, \proglang{R}}
\Plainkeywords{fixed income, term structure estimation, credit spreads, R} %% without formatting
%% at least one keyword must be supplied

%% publication information
%% NOTE: Typically, this can be left commented and will be filled out by the technical editor
%% \Volume{13}
%% \Issue{9}
%% \Month{September}
%% \Year{2004}
%% \Submitdate{2004-09-29}
%% \Acceptdate{2004-09-29}

%% The address of (at least) one author should be given
%% in the following format:
\Address{
  Robert Ferstl\\
  Institute for Operations Research\\
  Wirtschaftsuniversit\"at Wien\\
  1090 Vienna, Austria\\
  E-mail: \email{robert.ferstl@wu-wien.ac.at}\\
  URL: \url{http://www.wu-wien.ac.at/or}\\
  
   Josef Hayden\\
  Financial Engineering and Derivatives Group\\
  Wirtschaftsuniversit\"at Wien\\
  1090 Vienna, Austria\\
  E-mail: \email{josef.hayden@wu-wien.ac.at}\\
  URL: \url{http://www.wu-wien.ac.at/fed}\\
}
%% It is also possible to add a telephone and fax number
%% before the e-mail in the following format:
%% Telephone: +43/1/31336-5053
%% Fax: +43/1/31336-734

%% for those who use Sweave please include the following line (with % symbols):
%% need no \usepackage{Sweave.sty}

%% end of declarations %%%%%%%%%%%%%%%%%%%%%%%%%%%%%%%%%%%%%%%%%%%%%%%


\begin{document}

%% include your article here, just as usual
%% Note that you should use the \pkg{}, \proglang{} and \code{} commands.
%% Note: If there is markup in \(sub)section, then it has to be escape as above.

\section{Introduction}

Hello World!
\cite{BIS2005}
\section{Nelson/Siegel and Svensson method}
\label{sec:nels-svenss-meth}

\cite{Nelson1987} propose a parsimonious  model of  the instantaneous forward rate as a solution to a second-order differential equation for the case of equal roots.

\begin{equation}
  \label{eq:laguerre}
  f(m,\bm{b}) = \beta_0+\beta_1\exp\left(-\frac{m}{\tau_1}\right)+\beta_2\left[\left(\frac{m}{\tau_1}\right)\exp\left(-\frac{m}{\tau_1}\right)\right]
\end{equation}


The spot rate is the average of the instantaneous forward rates. 

\begin{equation}
  \label{eq:intspotrate}
  s(m,\bm{b})=\frac{1}{m}\int_0^mf(m,\bm{b})\,dm
\end{equation}


\begin{equation}
  \label{eq:nelson-spot}
   s(m,\bm{b}) = \beta_0 + \beta_1\frac{1-\exp(-\frac{m}{\tau_1})}{\frac{m}{\tau_1}} + \beta_2\left(\frac{1-\exp(-\frac{m}{\tau_1})}{\frac{m}{\tau_1}} - \exp(-\frac{m}{\tau_1})\right)
\end{equation}

\input{curveshape.tex}
 
\cite{Svensson1994} extension


\begin{multline}\label{eq:svensson-spot}
    s(m,\bm{b}) = \beta_0 + \beta_1\frac{1-\exp(-\frac{m}{\tau_1})}{\frac{m}{\tau_1}} + \beta_2\left(\frac{1-\exp(-\frac{m}{\tau_1})}{\frac{m}{\tau_1}} - \exp(-\frac{m}{\tau_1})\right) \\+ \beta_3\left(\frac{1-\exp(-\frac{m}{\tau_2})}{\frac{m}{\tau_2}} - \exp(-\frac{m}{\tau_2})\right)
\end{multline}



with a parameter vector ${\bm{b}} = \left(\beta_0,\beta_1,\beta_2,\tau_1,\beta_3,\tau_2\right)$.

The impact of the parameters can be described as follows \citep[see][p.7]{Bolder1999}:

\begin{itemize}
\item $\beta_0$ is the asymptotic value of the spot rate function $\lim_{m\to\infty}s(m,\bm{b})$, which can be seen as long-term interest rate.
\item $\beta_1$ is the limit of the spread between the spot rate function and the long term interest rate $\lim_{m\to\infty}\left[s(m,\bm{b})-\beta_0\right]$ determines the starting (short-term) value of the curve in terms of deviation from the asymptote (the sum $\beta_0$ and $\beta_1$) is the vertical intercept. Moreover, it defines the basic speed with which the curve tends toward its long-term trend.
\item $\beta_2$ determines the magnitude and direction of the hump. If $\beta_2 >0$  a hump will occur at  $\tau_1$, whereas $\beta_2<0$, a U-shaped value will occur at $\tau_1$.
\item $\tau_1$ is a scale parameter and specifies the position of the first hump or the U-shape on the curve ($\tau_1>0$).
\item $\beta_3$ analogously  to  $\beta_2$ determines the magnitude and direction of the second hump.
\item $\tau_2$ specifies the position of the second hump or the U-shape on the curve ($\tau_2>0$).
\end{itemize}


The discount factor for any maturity can be calculated as follows. 

\begin{displaymath}
d(m_{ij})=e^{-m_{ij}s(m_{ij},b)},
\end{displaymath}

where $s(m_{ij},b)$ is the Nelson/Siegel or Svensson spot rate function defined in \eqref{eq:nelson-spot} and \eqref{eq:svensson-spot}.

We optimize the objective function in \eqref{eq:optimalparam}. The above specification of the discount function leads to nonlinear optimization. Good starting values for the parameter vector are important to find a global minimum.



\cite{Soederlind1997} page 418 mentions that the equation for the yield has only one real root and is easy to solve numerically. They also have confidence intervals by the delta method.

This is a paramtric method \citep[see][chapter 15]{James2000}. Practioners favor smooth curve. Tradeoff with accuracy. Linear approximation is not appropriate.

\begin{itemize}
\item \cite{Geyer1999}
\item duration based weights \cite{Bliss1997}
\item derivation via Laguerre function
\item extensions \cite{Bjoerk 1999, Filipovic1999, Filipovic2000, Bjoerk2001, Bjoerk2002, Soederlind1997, Bliss1997}
\end{itemize}

%%% Local Variables: 
%%% mode: latex
%%% TeX-master: "jss-termstrc"
%%% End: 

\section{Cubic splines}
\label{sec:cubic-splines}




%%% Local Variables: 
%%% mode: latex
%%% TeX-master: "jss-termstrc"
%%% End: 

\subsection{Nelson/Siegel}

\begin{Schunk}
\begin{Sinput}
> library(termstrc)
\end{Sinput}
\end{Schunk}

\begin{Schunk}
\begin{Sinput}
> data(eurobonds)
> group <- c("GERMANY", "AUSTRIA", "ITALY")
> bonddata <- eurobonds
> matrange <- c(2, 12)
> method <- "Nelson/Siegel"
> fit <- "prices"
> weights <- "none"
> control <- list(eval.max = 1e+05)
> b <- matrix(c(0, 0, 0, 1, 0, 0, 0, 1, 0, 0, 0, 1), nrow = 3, 
+     ncol = 4, byrow = TRUE)
> rownames(b) <- group
> colnames(b) <- c("beta0", "beta1", "beta2", "tau1")
\end{Sinput}
\end{Schunk}

\begin{Schunk}
\begin{Sinput}
> x <- nelson_estim(group, bonddata, matrange, method, fit, weights, 
+     startparam = b, control)
> print(x)
\end{Sinput}
\begin{Soutput}
---------------------------------------------------
Parameters for Nelson/Siegel, Svensson estimation:

Method: Nelson/Siegel 
Fitted: prices 
Weights: none 

---------------------------------------------------

             GERMANY     AUSTRIA       ITALY
beta_0  0.0418991850  0.04167176  0.04645068
beta_1 -0.0190504299 -0.01716246 -0.02012130
beta_2 -0.0001155352 -0.01451567 -0.02134967
tau_1   3.9893303512  2.11920557  2.39845119

---------------------------------------------------
Parameters for Nelson/Siegel, Svensson estimation:

Method: Nelson/Siegel 
Fitted: prices 
Weights: none 

---------------------------------------------------

             GERMANY     AUSTRIA       ITALY
beta_0  0.0418991850  0.04167176  0.04645068
beta_1 -0.0190504299 -0.01716246 -0.02012130
beta_2 -0.0001155352 -0.01451567 -0.02134967
tau_1   3.9893303512  2.11920557  2.39845119
\end{Soutput}
\end{Schunk}

\begin{Schunk}
\begin{Sinput}
> summary(x)
\end{Sinput}
\begin{Soutput}
---------------------------------------------------
Goodness of fit:
---------------------------------------------------

                  GERMANY      AUSTRIA        ITALY
RMSE-Prices  1.059285e-01 2.868050e-02 0.0790745094
AABSE-Prices 6.157349e-02 2.000662e-02 0.0561561019
RMSE-Yields  1.324394e-04 4.051508e-05 0.0001240929
AABSE-Yields 8.941224e-05 2.994120e-05 0.0001016403


---------------------------------------------------
Convergence information:
---------------------------------------------------

        Convergence ()  
GERMANY "no convergence"
AUSTRIA "converged"     
ITALY   "converged"     

        Solver message                                   
GERMANY "iteration limit reached without convergence (9)"
AUSTRIA "relative convergence (4)"                       
ITALY   "relative convergence (4)"                       
\end{Soutput}
\end{Schunk}

\begin{center}
\begin{Schunk}
\begin{Sinput}
> par(mfrow = c(3, 2), cex = 0.55)
> plot(x)
\end{Sinput}
\end{Schunk}
\includegraphics{example-005}
\end{center}

\subsection{Cubic Splines}

\begin{Schunk}
\begin{Sinput}
> data(eurobonds)
> group <- c("GERMANY", "AUSTRIA", "ITALY")
> bonddata <- eurobonds
> matrange <- "all"
\end{Sinput}
\end{Schunk}

\begin{Schunk}
\begin{Sinput}
> x <- splines_estim(group, bonddata, matrange)
> print(x)
\end{Sinput}
\begin{Soutput}
---------------------------------------------------
Parameters for Cubic splines estimation:

[1] "GERMANY:"
      alpha 1       alpha 2       alpha 3       alpha 4       alpha 5 
-0.0031562638 -0.0001804021  0.0008893256  0.0009869789 -0.0235685086 

[1] "AUSTRIA:"
      alpha 1       alpha 2       alpha 3       alpha 4 
-0.0030226383  0.0007101622  0.0010939255 -0.0234513152 

[1] "ITALY:"
      alpha 1       alpha 2       alpha 3       alpha 4       alpha 5 
-1.319834e-03 -2.612472e-03 -3.871354e-05  1.264864e-03  6.204093e-04 
      alpha 6 
-2.481481e-02 
\end{Soutput}
\end{Schunk}


\begin{Schunk}
\begin{Sinput}
> summary(x)
\end{Sinput}
\begin{Soutput}
---------------------------------------------------
Goodness of fit:
---------------------------------------------------

                  GERMANY      AUSTRIA        ITALY
RMSE-Prices  1.818950e+01 1.386148e+01 2.283539e+01
AABSE-Prices 1.406844e+01 8.328159e+00 1.536100e+01
RMSE-Yields  1.224657e-04 1.114989e-04 1.499874e-04
AABSE-Yields 8.282076e-05 8.727466e-05 1.155357e-04

---------------------------------------------------
Summary statistics for the fitted models:
---------------------------------------------------

$GERMANY

Call:
lm(formula = -Y[[k]] ~ X[[k]] - 1)

Residuals:
      Min        1Q    Median        3Q       Max 
-0.133728 -0.043399 -0.009263  0.019054  0.337135 

Coefficients:
          Estimate Std. Error t value Pr(>|t|)    
alpha 1 -3.156e-03  4.765e-04  -6.623 7.51e-07 ***
alpha 2 -1.804e-04  1.572e-04  -1.148    0.262    
alpha 3  8.893e-04  4.782e-05  18.598 9.25e-16 ***
alpha 4  9.870e-04  6.148e-05  16.053 2.45e-14 ***
alpha 5 -2.357e-02  5.786e-04 -40.734  < 2e-16 ***
---
Signif. codes:  0 ‘***’ 0.001 ‘**’ 0.01 ‘*’ 0.05 ‘.’ 0.1 ‘ ’ 1 

Residual standard error: 0.1084 on 24 degrees of freedom
Multiple R-Squared:     1,	Adjusted R-squared:     1 
F-statistic: 1.693e+06 on 5 and 24 DF,  p-value: < 2.2e-16 


$AUSTRIA

Call:
lm(formula = -Y[[k]] ~ X[[k]] - 1)

Residuals:
       Min         1Q     Median         3Q        Max 
-0.0806338 -0.0430618  0.0006852  0.0263651  0.1077566 

Coefficients:
          Estimate Std. Error t value Pr(>|t|)    
alpha 1 -3.023e-03  1.274e-04  -23.72 4.03e-10 ***
alpha 2  7.102e-04  4.795e-05   14.81 3.95e-08 ***
alpha 3  1.094e-03  8.459e-05   12.93 1.44e-07 ***
alpha 4 -2.345e-02  2.255e-04 -104.01  < 2e-16 ***
---
Signif. codes:  0 ‘***’ 0.001 ‘**’ 0.01 ‘*’ 0.05 ‘.’ 0.1 ‘ ’ 1 

Residual standard error: 0.05827 on 10 degrees of freedom
Multiple R-Squared:     1,	Adjusted R-squared:     1 
F-statistic: 1.479e+06 on 4 and 10 DF,  p-value: < 2.2e-16 


$ITALY

Call:
lm(formula = -Y[[k]] ~ X[[k]] - 1)

Residuals:
      Min        1Q    Median        3Q       Max 
-0.268791 -0.051542 -0.003029  0.056274  0.162438 

Coefficients:
          Estimate Std. Error t value Pr(>|t|)    
alpha 1 -1.320e-03  1.827e-03  -0.722    0.476    
alpha 2 -2.612e-03  4.548e-04  -5.745 2.85e-06 ***
alpha 3 -3.871e-05  9.279e-05  -0.417    0.679    
alpha 4  1.265e-03  4.061e-05  31.149  < 2e-16 ***
alpha 5  6.204e-04  6.910e-05   8.978 5.29e-10 ***
alpha 6 -2.481e-02  1.180e-03 -21.029  < 2e-16 ***
---
Signif. codes:  0 ‘***’ 0.001 ‘**’ 0.01 ‘*’ 0.05 ‘.’ 0.1 ‘ ’ 1 

Residual standard error: 0.1005 on 30 degrees of freedom
Multiple R-Squared:     1,	Adjusted R-squared:     1 
F-statistic: 2.049e+06 on 6 and 30 DF,  p-value: < 2.2e-16 
\end{Soutput}
\end{Schunk}

\begin{center}
\begin{Schunk}
\begin{Sinput}
> par(mfrow = c(3, 2), cex = 0.55)
> plot(x)
\end{Sinput}
\end{Schunk}
\includegraphics{example-009}
\end{center}



\section{Conclusion}
\label{sec:conclusion}

In this paper, we presented the \proglang{R} extension package \pkg{termstrc}. It provides functions for the estimation of zero-coupon yield curves from market data of coupon bonds. The package covers the two most widely-used approaches in practice and provides a simple interface to them. The results contain detailed summaries about the estimation, as well as graphical outputs of spot, forward and credit spread curves.


\section*{Acknowledgments}

The authors want to thank Alois Geyer and Kurt Hornik for their comments about the package and the paper.


%%% Local Variables: 
%%% mode: latex
%%% TeX-master: "jss-termstrc"
%%% End: 


We use \cite{R2007}.

%\nocite{*}
\listoftables
\listoffigures
\bibliographystyle{jss}
\bibliography{termstrc}

\end{document}



%%% Local Variables: 
%%% mode: latex
%%% TeX-master: "jss-termstrc"
%%% End: 
