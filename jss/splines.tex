\newpage
\section{Cubic splines}
\label{sec:cubic-splines}

\cite{McCulloch1971, McCulloch1975} use the following definition of the discount function:

\begin{equation}
  \label{eq:df_spline}
  \delta(m_{ij},\bm{\beta})=1+\sum_{l=1}^k\beta^lg^l(m_{ij}).
\end{equation}

It is a linear combination of functions satisfying $g^l(0)=0$, and the unknown parameter vector $\bm{\beta}$ will be estimated with ordinary least squares (OLS). This piecewise function is twice-differentable at each knot point, which results in a smooth curve.

\subsection{Knot point selection}

\cite{McCulloch1975} defines a $k$-parameter spline with $k-1$ knot points $q_l$. We sort the cashflow matrix $\bm{C}$ and the maturity matrix $\bm{M}$ such that the $m$ bonds are arranged in ascending order by their maturity dates $\bm{m}$. The following specification places an approximately equal number of bonds between adjacent knots. It sets $q_1=0$ and $q_{k-1}=m_m$. $m_j$ is the maturity date of the $j$-th bond. For $1<l<k-1$ we find the further knot points:

\begin{equation*}
  \label{eq:A.1a}
  q_l = m_h+\theta(m_{h+1}-m_h),
\end{equation*}

where

\begin{equation*}
 \label{eq:A.1b}
  h = \left\lceil\frac{(l-1)m}{k-2}\right\rceil
\end{equation*}

and

\begin{equation*}
  \label{eq:A.1c}
  \theta = \frac{(l-1)m}{k-2}-h\,.
\end{equation*}

\cite{McCulloch1971} sets the  number of basis functions $k$ to the integer nearest to the square root of the number of observed bonds. 

\begin{equation*}
  \label{eq:nofknots}
  k = \left\lfloor\sqrt{m}+0.5\right\rfloor
\end{equation*}

This allows a smooth fit of the discount function.

\subsection{Basis functions for cubic splines}

In order to generate the family of cubic splines relative to these knots, we define for $m_{ij}<q_{l-1}$

\begin{equation*}
  \label{eq:A.2}
  g^l(m_{ij})=0\,.
\end{equation*}


For $q_{l-1}\leq m_{ij} <q_l$, we define

\begin{equation*}
  \label{eq:A.3}
  g^l(m_{ij})=\frac{(m_{ij}-q_{l-1})^3}{6(q_l-q_{l-1})}.
\end{equation*}

When $q_l\leq m_{ij}< q_{l+1}$, we define

\begin{equation*}
  \label{eq:A.4a}
  g^l(m_{ij}) = \frac{c^2}{6}+\frac{ce}{2}+\frac{e^2}{2}-\frac{e^3}{6(q_{l+1}-q_l)}\,,
\end{equation*}

where

\begin{equation*}
  \label{eq:A.4b}
  c = q_l-q_{l-1}
\end{equation*}

and $e$ is

\begin{equation*}
  \label{eq:A.4c}
  e = m_{ij}-q_l\,.
\end{equation*}

For $q_{l+1}\leq m_{ij}$, we define

\begin{equation*}
  \label{eq:A.5}
  g^l(m_{ij}) = (q_{l+1}-q_{l-1})\left[\frac{2q_{l+1}-q_l-q_{l-1}}{6}+\frac{m_{ij}-q_{l+1}}{2}\right]\,.
\end{equation*}

(Set $q_{l-1}=q_l=0$ when $l=1$.)

The above formulas apply when $l<k$. When $l=k$, we define

\begin{equation*}
  \label{eq:A.6}
 g^l(m_{ij})= m_{ij},
\end{equation*}

regardless of $m_{ij}$.

The basis functions are calulated for the maturities of each cashflow $m_{ij}$ and summarized in the matrix

\begin{equation*}
\label{eq:basisfctmatrix}	
\bm{G}_{\left [n \times m\right]}^l= \{g_{ij}^l\}\,,
\end{equation*}
where $g_{ij}^l=g^l(m_{ij})$.



\subsection{Regression fitting of the discount function}

The dirty prices are again expressed as the sum of the discounted cashflows plus an idiosyncratic error

\begin{equation}
  \label{eq:fairpricespline}
  \bm{p}= \bm{\iota}'\left(\bm{C}\cdot\bm{D}\right)+ \bm{\epsilon}.
\end{equation}

The discount factor matrix is defined as the weighted sum of the $l=1\dots k$ basis functions

\begin{equation}
  \label{eq:dfmatrixspline}
   \bm{D}=1+\beta^1\bm{G}^1+\cdots + \beta^k\bm{G}^k.
\end{equation}

We substitute \eqref{eq:dfmatrixspline} in \eqref{eq:fairpricespline} and get an expression which is linear in the parameter vector $\bm{\beta}=(\beta^1,\dots,\beta^k)$.

\begin{eqnarray*}
  \label{eq:splineparam}
  \bm{p} &=& \bm{\iota}'\left(\bm{C}\cdot\left(1+\beta^1\bm{G}^1+\cdots+\beta^k\bm{G}^k\right)\right)+\bm{\epsilon}\\
  \bm{p} &=& \bm{\iota}'\left(\bm{C}+\bm{C}\cdot\left(\beta^1\bm{G}^1+\cdots+\beta^k\bm{G}^k\right)\right)+\bm{\epsilon}\\
  \bm{p} &=& \bm{\iota}'\bm{C}+\bm{\iota}'\bm{C}\cdot\left(\beta^1\bm{G}^1+\cdots+\beta^k\bm{G}^k\right)+\bm{\epsilon}\\
  \bm{p}- \bm{\iota}'\bm{C} &=& \beta^1\bm{\iota}'\bm{C}\cdot\bm{G}^1+\cdots+\beta^k\bm{\iota}'\bm{C}\cdot\bm{G}^k+\bm{\epsilon}
\end{eqnarray*}


We summarize the terms on both sides as follows:


\begin{equation*}
\bm{X}_{\left[m \times k\right]}=\{ \bm{x}_{\left[m \times 1\right]} \} \qquad  \bm{x}_{\left[m \times 1\right]} = \left( \bm{\iota}'\bm{C}\cdot\bm{G}^l\right)'
\end{equation*}


\begin{equation*}
\bm{z}_{\left[m \times 1\right]}= \left(\bm{p}-\bm{\iota}'\bm{C}\right)'
\end{equation*}
       

\begin{equation}
\label{eq:splinereg}
    \bm{z}=\bm{X}\bm{\beta}+\bm{\epsilon}
\end{equation}
          
The unknown parameters can now be estimated with OLS.
	
\begin{equation*}
\label{eq:paramspline}
\bm{\hat \beta}_{\left[k\times 1\right]}= \left( \bm{X}' \bm{X}\right )^{-1}\bm{X}' \bm{z}
\end{equation*}

We can use the resulting parameters to calculate the discount function in \eqref{eq:df_spline} for any given maturity $m_{ij}$ between the first and the last knot point, which can then be converted to the spot rate function

\begin{equation*}
  \label{eq:transformdf}
  s(m_{ij},\bm{\beta}) = \frac{-\ln \delta(m_{ij},\bm{\beta})}{m}.
\end{equation*}


\subsection{Confidence intervals for the discount function}

\cite{McCulloch1975} plots error bands one standard error above and below of the best estimate. We derive a confidence interval for the predicted discount function. Under the assumption of normally distributed disturbances $\bm{\epsilon}$, the ordinary least squares coefficient estimator of \eqref{eq:splinereg} is normally distributed with mean $\bm{\beta}$ and variance-covariance matrix $\sigma^2\bm{X}'\bm{X}^{-1}$. 

Following \cite{Greene2002}, the confidence interval for a linear combination of coefficients can be obtained by applying the decomposition of \cite{Oaxaca1973}. Therefore, the discount function $\delta(m_{ij})$ in \eqref{eq:df_spline} is normally 
distributed with mean
\begin{equation*}
\mu = 1+\bm{g}(m_{ij})'\bm{\beta}
\end{equation*}

and variance

\begin{equation*}
\sigma^2 = \bm{g}(m_{ij})'\left(\sigma^2(\bm{X' X})^{-1} \right)\bm{g}(m_{ij}),
\end{equation*}

where $\bm{g}(m_{ij})= \left(g^1(m_{ij}), \dots, g^l(m_{ij}) \right)$.

The $1- \alpha$ confidence interval for $\mu$ of the discount function can now be constructed in the usual way

\begin{equation*}
\label{eq:cint}
\mbox{P}\left[ \delta(m_{ij}) - t_{\alpha / 2} s   \leq \mu  \leq \delta(m_{ij})  + t_{\alpha / 2} s\right]= 1 - \alpha,
\end{equation*} 
where $s$ is the estimate for $\sigma$, $t_{\alpha / 2}$ the appropriate critical value from the $t$-distribution with $m-k$ degrees of freedom and $1-\alpha$ the desired level of confidence. 


%%% Local Variables: 
%%% mode: latex
%%% TeX-master: "jss-termstrc"
%%% End: 
