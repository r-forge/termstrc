\section{Software implementation}

\subsection{Data sets}

We use the following list structure for a data set of coupon bonds. It contains all the necessary information to perform an estimation of the zero-coupon yield curve.

\begin{table}[htb]
  \centering
  \begin{tabular}[htb]{ll}
    \textbf{Object} & \textbf{Description} \\
ble & bla
  \end{tabular}
  \caption{Contents of data set}
\end{table}

 $ GERMANY:List of 8
  ..$ ISIN        : chr [1:52] "DE0001141414" "DE0001137131" "DE0001141422" "DE0001137149" ...
  ..$ MATURITYDATE:Class 'Date'  num [1:52] 13924 13952 13980 14043 14064 ...
  ..$ ISSUEDATE   :Class 'Date'  num [1:52] 11913 13215 12153 13298 10411 ...
  ..$ COUPONRATE  : num [1:52] 0.0425 0.0300 0.0300 0.0325 0.0413 ...
  ..$ PRICE       : num [1:52] 100.0  99.9  99.8  99.8 100.1 ...
  ..$ ACCRUED     : num [1:52] 4.09 2.66 2.43 2.07 2.39 ...
  ..$ CASHFLOWS   :List of 3
  .. ..$ ISIN: chr [1:384] "DE0001141414" "DE0001137131" "DE0001141422" "DE0001137149" ...
  .. ..$ CF  : num [1:384] 104 103 103 103 104 ...
  .. ..$ DATE:Class 'Date'  num [1:384] 13924 13952 13980 14043 14064 ...
  ..$ TODAY       :Class 'Date'  num 13908


%%% Local Variables: 
%%% mode: latex
%%% TeX-master: "jss-termstrc"
%%% End: 
