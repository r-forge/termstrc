\documentclass[a4paper,10pt]{elsart}
%\usepackage{a4wide}
\usepackage[ansinew]{inputenc}
\usepackage{graphicx}

\usepackage[bookmarks=true, bookmarksopen=true,
                bookmarksnumbered=true, colorlinks,citecolor = blue,
                filecolor=blue, linkcolor=red, urlcolor=red,
                plainpages=false,hyperindex=true]{hyperref}
\hypersetup{
        pdftitle={Cash Management using Multi-Stage Stochastic Programming},
        pdfauthor={Robert Ferstl, Alex Weissensteiner},
        pdfsubject={},
        pdfkeywords={},
        pdfcreator={},
        pdfproducer={}
     }

\usepackage{natbib}
\usepackage{amsmath,amstext}
%\usepackage{manfnt}

\setlength{\parindent}{0.0cm}                      % Absatzeinr�ckungen
\setlength{\parskip}{1.5ex plus 0.5ex minus 0.5ex} % Absatzabst�nde
%\renewcommand{\arraystretch}{1.5}                 % Zeilenabstand in Tabellen

\begin{document}
%\title{Cash Management using\\ Multi-Stage Stochastic Programming}
%\author{Robert Ferstl and Alex Weissensteiner}
% \maketitle

\begin{frontmatter}
\title{Term Structure and Credit Spread\\Estimation with R}


\author[label1]{Robert Ferstl}
\address[label1]{Institute for Operations Research,\\Vienna University of Economics and Business Administration}
%\ead{robert.ferstl@wu-wien.ac.at}

\author[label2]{Josef Hayden}
\address[label2]{Financial Engineering and Derivatives Group,\\Vienna University of Economics and Business Administration}
%\ead{josef.hayden@wu-wien.ac.at}

\begin{keyword}
fixed income, term structure estimation, credit spreads
\end{keyword}
\end{frontmatter}

Zero-coupon yield curves and credit spread curves are important inputs for various financial models, e.g. pricing of securities, risk management, monetary policy issues. Since zero-coupon rates are rarely directly observable, they have to be estimated from market data, e.g. of existing coupon bonds. The literature broadly distinguishes between parametric and spline-based estimation methods for the zero-coupon yield curve. Our package consists of several widely-used approaches, i.e. the parametric \cite{Nelson1987} method with the \cite{Svensson1994} extension, and the \cite{McCulloch1971} cubic splines approach. Moreover, we implement the traditional way of credit spread calculation, where individually estimated zero-coupon yield curves are substracted from a risk-free reference curve. Goodness-of-fit tests are provided to compare the results of the different estimation methods. We illustrate the usage of our functions by practical examples with data from European and CEE government bonds, and European corporate bonds.



%------------------------------------------------------
%input "termstrc.bib"
\markboth{References}{References} \frenchspacing
\bibliographystyle{apalike}
\bibliography{termstrc}
\end{document}

%%% Local Variables:
%%% mode: latex
%%% TeX-master: t
%%% End:
