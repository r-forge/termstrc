\documentclass[mathserif,10pt]{beamer}
\hypersetup{pdfmode=FullScreen}
%\usepackage{pgfpages}
%\usepackage{qtree}

\mode<presentation>
{
  \usetheme{shadow}
  \usecolortheme{crane}
  \setbeamercovered{transparent}
  \useoutertheme{infolines}
}

\usepackage{times}
\usepackage{graphicx}
\usepackage[T1]{fontenc}
\usepackage{amssymb,amsmath,amsthm, amsbsy, bm,dsfont}
\newcommand{\bs}{\boldsymbol}
\newcommand{\dd}{\mathrm{d}}
\beamertemplatenavigationsymbolsempty
\DeclareMathOperator{\maximize}{maximize}
\DeclareMathOperator{\E}{\mathbb{E}}
\DeclareMathOperator{\V}{V\mathbb{V}}
\DeclareMathOperator{\VaR}{VaR}
\DeclareMathOperator{\CVaR}{CVaR}




%for R Syntaxhighlighting
\usepackage{listings}
\usepackage{color}


\lstdefinelanguage{R}%
  {keywords={c,list, matrix, nrow,ncol,rownames,colnames,nelson_estim, print, summary, plot,byrow},%
   otherkeywords={!,!=,~,$,*,\&,\%/\%,\%*\%,\%\%,<-,<<-,/},%
   alsoother={._$},%
   sensitive,%
   morecomment=[l]\#,%
   morestring=[d]",%
   morestring=[d]'% 
  }







\title[Term Structure and Credit Spread Estimation with R]{Term Structure and Credit Spread \\Estimation with R}

\author[Robert Ferstl\and Josef Hayden]{Robert Ferstl\inst{1} \and Josef Hayden\inst{2}}


\institute[]{
\inst{1}Institute for Operations Research\\
Vienna University of Economics and Business Administration
\and
\inst{2}Financial Engineering and Derivatives Group\\
Vienna University of Economics and Business Administration}

\date[AWG]{\scriptsize\em \textbf{First R/Rmetrics User and Developer Workshop\\Meielisalp, Lake Thune, Switzerland\\\vspace{0.2cm} July 8-12, 2007}}

\pgfdeclareimage[height=1.0cm]{wu-logo}{wu-logo}
%\logo{\pgfuseimage{wu-logo}}	% show logo

%\setbeameroption{show notes}
\setbeameroption{hide notes}


\begin{document}
%define colors for syntax highlighting
\definecolor{darkblue}{rgb}{0,0,.5}
\definecolor{darkred}{rgb}{0.4,0.0,0}
\definecolor{commentcolor}{rgb}{0,0.5,0.25}
\definecolor{stringcolor}{rgb}{0.5,0.0,0}

\definecolor{myblue}{rgb}{.8, .8, 1}
\newcommand*\mybluebox[1]{%
\colorbox{myblue}{\hspace{1em}#1\hspace{1em}}}


%settings for R syntaxhighlighting
\lstset{basicstyle=\ttfamily\small,
keywordstyle=\bfseries\color{darkblue},
stringstyle=\ttfamily,
showstringspaces=false,
language=R,
identifierstyle=,
commentstyle=\color{commentcolor},
stringstyle=\color{stringcolor}}





\frame{\titlepage}

\section<presentation>*{Outline}

\begin{frame}
  \frametitle{Outline}
  \tableofcontents[pausesections]
\end{frame}

\section{Introduction}

\subsection{Basic Principles of Bond Pricing}

\begin{frame}
  \frametitle{Basic Principles of Bond Pricing}
  \begin{itemize}
    \item coupon bond which matures in $n$ years
    \item investor gets cashflows $c_t$ at the times $t=1,\dots n$ ($c_n$ includes the redemption payment)
      \item \textcolor{craneblue}{\textbf{clean price}} $p_c$ is quoted on the market
    \item seller also receives \textcolor{craneblue}{\textbf{accrued interest}} for holding the bond over the period since the last coupon payment
  	\begin{equation*}
  \label{accruedinterest}
  a=\frac{\mbox{number of days since last coupon}}{\mbox{number of days in current coupon period}}C
\end{equation*}
	\item investor has to pay the \textcolor{craneblue}{\textbf{dirty price}} $p_d$
\item bond pricing equation with continuous compounding
\begin{equation*}
  \label{bondpriceeq}
  p_c+a = \sum_{t=1}^n \ c_t e^{-s_tm_t}
\end{equation*}
\end{itemize}
\end{frame}

\begin{frame}
\frametitle{Basic Principles of Bond Pricing}
  \begin{itemize}
\item \textcolor{craneblue}{\textbf{yield to maturity}}
\begin{equation*}
  \label{yield}
   p_c+a=\sum_{t=1}^n \ c_t e^{-ym_t}
\end{equation*}
\item  equivalent formulation of the bond price equation uses the \textcolor{craneblue}{\textbf{discount factors}} $d_t=\delta(m_t)=e^{-s_tm_t}$
\item continuous \textcolor{craneblue}{\textbf{discount function}} $\delta(\cdot)$ is formed by interpolation of the discount factors
  \begin{equation*}
  \label{bondprceq2}
    p_c+a=\sum_{t=1}^n \ c_t \delta(m_t) 
  \end{equation*}
   \item \textcolor{craneblue}{\textbf{duration}} is a weighted average of time to cash flows
  \begin{equation*}
 \label{duration} D=\frac{1}{p_c+a}\left[C\sum_{i=1}^n\delta(m_i)m_i+\delta(m_n)Rm_n\right]
\end{equation*}
\end{itemize}
\end{frame}

\subsection{Term Structure Estimation}

% show data set
% data(eurobonds)
% str(eurobonds)

\begin{frame}
	\frametitle{Term Structure Estimation Procedure \newline Notation I}
	\begin{beamerboxesrounded}[shadow=true]{Maturity matrix $\bm{M}$}
		\begin{equation*}\label{maturitym}
		\bm{M}_{\left[n\times m\right]} = \{m_{ij}\}
		\end{equation*}
	\end{beamerboxesrounded}
	
	\vspace{0.3cm}
	\begin{beamerboxesrounded}[shadow=true]{Cashflow matrix $\bm{C}$}
		 \begin{equation*}\label{cashflowm}	
		\bm{C}_{\left[n\times m\right]} = \{c_{ij}\}
		\end{equation*}
	\end{beamerboxesrounded}
	
	\vspace{0.3cm}
	\begin{beamerboxesrounded}[shadow=true]{Clean price vector $\bm{p}^c$}
		 \begin{equation*}\label{pc}
		\bm{p}^c_{\left[1\times m\right]} = \{p^c_j\}
		\end{equation*}
	\end{beamerboxesrounded}	
	
	\vspace{0.3cm}
	\begin{beamerboxesrounded}[shadow=true]{Accrued interest vector $\bm{a}$}
		\begin{equation*}\label{a}
		\bm{a}_{\left[1\times m\right]} = \{a_j\}
		\end{equation*}
	\end{beamerboxesrounded}
	
\end{frame}


\begin{frame}
	\frametitle{Term Structure Estimation Procedure \newline Notation II}
	
	\vspace{0.2cm}
	\begin{beamerboxesrounded}[shadow=true]{Dirty price vector $\bm{p}^d$}
		\begin{equation*}\label{pd}
   		\bm{p}^d_{\left[1\times m\right]}= \{p^d_j\}
		\end{equation*}
		\begin{displaymath}
		\bm{p}^d=\bm{p}^c+\bm{a}
		\end{displaymath}
	\end{beamerboxesrounded}
	
	\vspace{0.2cm}
	\begin{beamerboxesrounded}[shadow=true]{Discount factor matrix $\bm{D}$}
		\begin{equation*}\label{discountm}
		\bm{D}_{\left[n\times m\right]} = \{d_{ij}\}; \qquad d_{ij}=e^{-m_{ij}s(m_{ij},\bm{b})}
		\end{equation*}
	\end{beamerboxesrounded}	
	
	\end{frame}

\section{Nelson/Siegel and Svensson Method}
\subsection{Notation}
\begin{frame}
	\frametitle{Spotrate Functions}
	\begin{block}{Nelson/Siegel:}
	\begin{equation*}\label{nelson}
   	 s(m,\bm{b}) = \beta_0 + \beta_1\frac{1-\exp(-\frac{m}{\tau_1})}{\frac{m}{\tau_1}} + \beta_2\left(\frac{1-\exp(-\frac{m}{\tau_1})}{\frac{m}{\tau_1}} - 	\exp(-\frac{m}{\tau_1})\right)
	\end{equation*}
	\end{block}
	
	\begin{block}{Svensson:}
	\begin{multline*}\label{svspot}
  	  s(m,\bm{b}) = \beta_0 + \beta_1\frac{1-\exp(-\frac{m}{\tau_1})}{\frac{m}{\tau_1}} + \beta_2\left(\frac{1-\exp(-\frac{m}{\tau_1})}{\frac{m}{\tau_1}} - 	\exp(-\frac{m}{\tau_1})\right) \\+ \beta_3\left(\frac{1-\exp(-\frac{m}{\tau_2})}{\frac{m}{\tau_2}} - \exp(-\frac{m}{\tau_2})\right)
	\end{multline*}
	\end{block}
\end{frame}


\begin{frame}
	\frametitle{Minimisation of the weighted pricing errors}
	\vspace{0.2cm}
	\begin{beamerboxesrounded}[shadow=true]{Weights vector $\bm{w}$}
		\begin{equation*}\label{weights}
   		\bm{w}_{\left[1\times m\right]}= \{w_j\}; \qquad   w_j=\frac{\frac{1}{D_j}}{\sum_{i=1}^m\frac{1}{D_i}}
		\end{equation*}
	\end{beamerboxesrounded}	
	
	\begin{block}{Objective function}
	\begin{equation*}
	F= \left(\left(\bm{\iota}_{\left[1 \times n\right]}\left[\bm{C}\cdot\bm{D}\right] - 		\bm{p}^d\right)^2 \bm{w}\bm{\iota}_{\left[m \times 1\right]} \right)
		 \end{equation*}
	\end{block}
    	\begin{block}{Minimisation problem}
	 $$\min_{\bm{b} \in G} F(\bm{b}) $$
	 $$ \bm{b} \in G =   \{ \bm{b} \in \mathds{R}^4 ; \mathds{R}^6 : \mathbf{lb} \leq \bm{b} \leq \mathbf{ub}  \}$$
	\end{block}
	
	\begin{block}{Parameter constraints}
	$$\beta_0 >0, \tau_1>0, \tau_2>0$$
	\end{block}
	

\end{frame}


\subsection{Example}

\begin{frame} [containsverbatim]\frametitle{Example}
  % \beamergotobutton{boxmuller}

\begin{lstlisting}
data(eurobonds)

group <- c("GERMANY", "AUSTRIA", "ITALY")
bonddata <- eurobonds
matrange <- c(2,10)
method <- "Nelson/Siegel"
fit <- "prices"
weights <- "none"
control <- list(eval.max=100000)

b <- matrix(c(0.02, -0.01, -0.025,  1,
 	0.026, -0.011, -0.0262,    1,
	0.025, -0.015, -0.025,    1),
	nrow=3,ncol=4,byrow=TRUE)
			
rownames(b) <- group

colnames(b) <- c("beta0","beta1","beta2","tau1")
x <- nelson_estim(group, bonddata, matrange, 
                  method, fit, weights, startparam=b,control)
\end{lstlisting}
\end{frame}


\section{Cubic Splines}
\subsection{Notation}

\begin{frame} 
\frametitle{Cubic splines }

\begin{itemize}
\item introduced by McCulloch (1975)
\item term structure is divided in segments
\item cubic functions are used to fit the term structure
\end{itemize}

\begin{beamerboxesrounded}[shadow=true]{Discount factor matrix $\bm{D}$}
		\begin{equation*}
		\bm{D}_{\left[n\times m\right]} = \{d_{ij}\}; \qquad d_{ij}= 1 + \sum_{l=1}^{s}\alpha_l g_{ij}(m_{ij},l)
		\end{equation*}
	\end{beamerboxesrounded}	




\end{frame}





 



\begin{frame}

	\frametitle{Term Strucuture Estimation \\ Notation III}
	%basic relationsship !!
	\begin{block}{Cubic functions matrix}
	$$\bm{G}_{\left[n \times m\right]} = \{  g_{ij}(m_{ij},l) \} \qquad   l = 1\dots s ; s= INT[\sqrt{m}]$$
	\end{block}
	
	\begin{block}{Predictor matrix}
	$$\bm{X}_{\left[m \times s\right]}=\{ \bm{x}_{\left[m \times 1\right]} \} \qquad  \bm{x}_{\left[m \times 1\right]} = \left( \iota_{\left[1\times n\right]} \bm{C}_{\left[n\times m\right]} \cdot \bm{G}_{\left[n\times m\right]}(l) \right)^t$$
	\end{block}
	
	\begin{block}{Response vector}
	$$\bm{y}_{\left[m \times 1\right]}=  \bm{p}^d_{\left[1\times m\right]}  - \iota_{\left[1\times n\right]} \bm{C}_{\left[n\times m\right]}   $$
	\end{block}
	
	\begin{block}{Parameter vector}
	$$\bm{\hat \alpha}_{\left[s \times 1\right]}= \left( \bm{X}^t   \bm{X}\right )^{-1}\bm{X}^t \bm{y}$$
	\end{block}
	
	


\end{frame}


\begin{frame} [containsverbatim]
	\frametitle{Example}
	\begin{lstlisting}
	data(eurobonds)

	group <- c("GERMANY", "AUSTRIA", "ITALY")
	bonddata <- eurobonds
	matrange <- "all"  

	x <- splines_estim(group, bonddata, matrange)

	print(x)
	summary(x)
	plot(x)
	\end{lstlisting}
	
\end{frame}


\section{Conclusion}
\begin{frame}\frametitle{Conclusion}

\begin{itemize}
\item  \textbf{termstrc} includes the two widley used term structure estimation methods
\item  two data sets and demo examples are provided
\item the package offers the common S3 methods
\item already available at \textbf{CRAN} !

\end{itemize}
\end{frame}

\begin{frame}
\Large \begin{center}Thank You for Your Attention !! \end{center}
\end{frame}





\begin{frame}[allowframebreaks]
 \frametitle<presentation>{References}

  \begin{thebibliography}{10}

  \scriptsize
  \beamertemplatearticlebibitems
  
  





%Charles R. Nelson and Andrew F. Siegel (1987): Parsimonious modeling of yield curves. The Journal of Business, 60(4):473�489.

%Lars E.O. Svensson (1994): Estimating and interpreting forward interest rates: Sweden 1992 -1994. Technical Reports 4871, National Bureau of Economic Research.
  
  
  
   \bibitem{BIS2005}
   Bank for International Settlements
   \newblock Zero-coupon yield curves: technical documentation
   \newblock {\em BIS Papers}, No. 25, October 2005
   
   \bibitem{Bliss2007}
   Robert R. Bliss 
   \newblock Testing term structure estimation methods. 
   \newblock {\em Advances in Futures and Options Research}, No. 9, p. 197--232 , 2007
  
  \bibitem{Bolder1999}
   David Bolder, David Streliski
   \newblock Yield Curve Modelling at the Bank of Canada
   \newblock {\em Bank of Canada, Technical Report}, No. 84, 1999
  
    \bibitem{Geyer1999}
   Alois Geyer, Richard Mader
    \newblock Estimation of the Term Structure of Interest Rates - A Parametric Approach
    \newblock {\em OeNB, Working Paper}, No. 37, 1999

  \bibitem{Jankowitsch2004}
    Rainer Jankowitsch, Stefan Pichler
    \newblock Parsimonious Estimation of Credit Spreads
    \newblock {\em The Journal of Fixed Income}, 14(3):49--63, 2004
    
    \bibitem{Ioannadis2003}
    Michalis Ioannides
    \newblock A comparison of yield curve estimation techniques using UK data.
    newblock {\em Journal of Banking \& Finance}, 27, 1--26
    
    \bibitem{McCulloch1971}
    J. Huston McCulloch
    \newblock Measuring the Term Structure of Interest Rates
    \newblock {\em The Journal of Business}, 44, 19--31
    
     \bibitem{McCulloch1975}
    J. Huston McCulloch
    \newblock  The Tax-Adjusted Yield Curve
    \newblock {\em The Journal of Finance}, 30, 811--830
    
    \bibitem{Nwalkhaetal2005}
     Sanjay K. Nawalkha and Gloria M. Soto and Natalia K. Beliaeva
     \newblock Interest Rate Risk Modeling
     \newblock{\em The Fixed Income Valuation Course}, Wiley Finance 2005
      
    \bibitem{Nelson1987}
    Charles R. Nelson, Andrew F. Siegel
    \newblock Parsimonious Modeling of Yield Curves
    \newblock {\em The Journal of Business}, 60(4):473--489, 1987
    
    \bibitem{Svensson1994}
    Lars E.O. Svensson
    \newblock Estimating and Interpreting Forward Interest Rates:\\Sweden 1992 -1994
    \newblock {\em National Bureau of Economic Research, \\Technical Report}, No. 4871, 1994

  \end{thebibliography}
\end{frame}

\end{document}
